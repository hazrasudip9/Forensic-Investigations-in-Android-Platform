
This algorithm is written in pseudo code and uses specially defined
math symbols.

\begin{figure}[b]
\begin{minipage}{.5\textwidth}
\begin{algorithm}[H]\small
  $K_1  := x := 0$\;
  \ForAll {$A \sqsubseteq B \in \OMath_1$} {
    UN \bang update$(B_{U}, A)$ \query $x$\;
    $K_1 += x$\;
  }
~\\[4mm]
  \caption{\small R1: $A \sqsubseteq B \Rightarrow U[B] \sqsubseteq U[A]$}
  \label{pseudo-R1-2}
\end{algorithm}
\end{minipage}
\textcolor{Gray}\quad\textcolor{Gray}{\vrule}\quad
\begin{minipage}{.5\textwidth}
\begin{algorithm}[H]\small
  $K_2 := x := 0$\;
  \ForAll {$A_1 \sqcap\dots\sqcap A_n\sqsubseteq B \in \OMath_2$} {
    UN \bang $\sqcap(B_{U}, \{A_1, \dots, A_n\})$ \query $x$\;
    $K_2 += x$\;
  }
  \label{pseudo-R2-2}
  \caption{\small R2: $A_1 \sqcap\dots\sqcap A_n\sqsubseteq B \Rightarrow
    U[B] \cupEq U[A_1] \cap \dots \cap U[A_n]$}
\end{algorithm}
\end{minipage}
\caption{Two Algorithms Side-by-Side}
\end{figure}

test line
\label{TwoAlgs}
