\newpage
\thispagestyle{plain}

{\centering\bf ABSTRACT\\}\par\vskip 2cm

\singleSpacing

\noindent
Iyer, Kuppu Samy. %% last, first name, Title Case
M.Tech.  Department of Cyber Security,
Amrita Vishwa Vidyapeetham,
2017.                           %% this year
{\sl Proactive Forensic Support for Android Devices} %% title

\par\vskip 1cm

\doubleSpacing

The advanced features of smartphones have attracted the attention of
the criminals.  Information found in a smart phone plays a crucial
role in the course of a criminal investigation and while presenting it
in a court of law.  An ordinary man in an ordinary pre-smart mobile
device easily gives away his/her contact details, call history and
SMS.  Smartphones contain even more useful information such as social
network messages, E-mails, network connections, browser history,
location history etc.  Such a device in the hands of a criminal is
assisting in his activities, but can also be helpful to a forensic
investigator.  But, a computer savvy criminal might modify or wipe the
traces of mobile data.  This data is harder to retrieve once the user
deletes it.  We designed proactive forensic support for the Android
ROM, which monitors all activities and stealthily and
oppurtunistically stores it in the cloud.

[A thesis abstract is a technical summary of the thesis.  It is
generally two-thirds of a page.  It should never contain citations.
The example above makes the mistake of including many sentences on
motivation, but very little on what that thesis contributed.  Of
course, the above is not an abstract of this template; the following
is.]

This report is a template for theses, including proposals.  It
explains the structure (chapters, sections, citations, bibiography,
etc.), and describes how to do various things (in particular, figures,
tables, algorithms and source code listings) in \LaTeX{}.  Pages that
are specific to Amrita Vishwa Vidyapeetham are located in {\tt
AmritaU/} directory.  The body of each chapter appears as a single
file in {\tt Chapers/} directory.

There is a separate companion template for papers with the typical
double-column single-spacing format.

It also has commentary on English language troubles (spelling, ...,
and in general on grammar and style) that non-native speakers of
English typically have.  A thesis writer should request some one who
is good at English to proof reed the thesis -- atleast the final
draft.  This is not a job of the advisor/ guide of the thesis.
Developing a good command over English is a life-long skill that will
be of great help in a student's career.

This template uses American spelling.

\vfill

\noindent
{\bf Keywords:} Android, Forensics, Digital Evidence, Android
Framework, {\tt adb} tool, Pocket Spy.

\newpage




