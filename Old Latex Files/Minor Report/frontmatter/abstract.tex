%
% File: abstract.tex
% Author: Gayathri Manikutty
% Description: Contains the text for thesis abstract
%
% UoB guidelines:
%
% Each copy must include an abstract or summary of the dissertation in not
% more than 300 words, on one side of A4, which should be single-spaced in a
% font size in the range 10 to 12. If the dissertation is in a language other
% than English, an abstract in that language and an abstract in English must
% be included.

\chapter*{Abstract}
\begin{SingleSpace}
\initial{S}martphones nowadays are capable of doing a multitude of tasks which was not possible with conventional phones. The capabilities which makes smartphones different from conventional phones is giving nightmares to security organisations in tracing and extracting data from the smartphones.  The criminals are now  able to wipe out the traces of criminal activities performed using their phones with much ease than before.  The criminal organisations are also using smartphones to communicate with others using encrypted messages which are very hard to trace and decrypt.  Even if the phones are confiscated after the crime, It is becoming very hard for law-enforcement agencies to extract data from those devices as these devices are encrypted with advanced encryption features and any attempt to get access to the device memory using brute-force would potentially wipe the all the data hence a proactive forensic framework was necessary for continuous monitoring of the criminals using their smartphones. The Proactive Forensic Rom is capable of monitoring the  online and offline activities of the suspected criminals with high degree of precision and stealth while discretely uploading the user data to the cloud for further monitoring and analysis purpose.
\end{SingleSpace}
\clearpage