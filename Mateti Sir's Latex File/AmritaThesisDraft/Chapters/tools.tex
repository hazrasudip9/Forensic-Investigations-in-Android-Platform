\chapter{Recommended Software Tools}

Only open-source tools are mentioned in this section.  Only a few
links are given below because they have been changing too rapidly.
Search the web.

\begin{description}

\item[{pdflatex}] directly produces .pdf files.  However, you may wish
  to produce .dvi, then .ps, and finaly .pdf.  DVI-to-PS is by dvips,
  PS to PDF is by ps2pdf.

\item[{qpdfview}], kdvi, xdvi, xpdf, gv, ghostview and okular are for
  previewing.

\item[tetex]There are multiple distributions of \TeX{} and \LaTeX{}
  for Linux.  I used tetex. {MiKTeX} is a complete \TeX{} distribution
  for Windows.

\item[lyx], {\tt texmacs} and {\tt kile} are frontends to \LaTeX{}
  giving a WYSIWIG view.  You might like them.

\item[{emacs}] is excellent for editing TeX files.  There are
  style files for TeX.  Make sure they are loading when you begin
  editing a file with the extension .tex.  Visit
  \url{http://www.emacswiki.org/}.

\item[tikz] Visit \url{http://www.texample.net/tikz/examples/} to see
  the many beautiful, and meticulous, vector drawings.  Steep
  learning.  So is \url{https://inkscape.org/en/}.

\item[xfig] is an old, but still good, GUI tool for vector-based
  drawing.  It can export into \LaTeX{}, eps and other formats.

\item [OpenOffice] and {\tt Inkscape} have vector drawing components
  that are quite good.  They can export into eps or pdf.

\item [{metapost},] {\tt asymptote} are programming languages for
  meticulous drawing.

\item[{graphviz}] is a layout tool for graph theoretic graphs.

\item[{tth}] generates {\sc html} files from your LaTeX files.
  The {tex2page} is better but requires the programming language
  Scheme.

\end{description}

% -eof-
