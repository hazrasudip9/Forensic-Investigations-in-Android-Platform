\chapter{English}

To most of our students English is not their mother tongue.  A thesis
writer should request some one who is good at English to proof reed
the thesis -- atleast the final draft.  This is not a job of the
advisor/ guide of the thesis.  Request a professor from the English
department to be your reader.

\section{Spelling}

In any technical report, spelling mistakes cannot be forgiven.
We use American spelling.  Watch for: -ise in verbs (should be -ize).
Spell checking LaTeX files can be awkward, but you must.  Doing it
from within Emacs is considerably better.


\section{Misused Words and Phrases}

We can turn a blind eye to an occasional
misuse of (i) the definite vesrsus indefinite article, and (ii) the
singular/plural confusion.

The following are collected from the theses drafts of my students.

\begin{enumerate}

\item Use the upper and lower case properly with names of software,
  etc. E.g., "Java" not "java", "Android" not "android".

\item
  The following words are almost always unnecessary: ``mainly'',
  ``basically'', ``very''.  Grep for each occurrence and remove.

\item
  A comma (almost) always follows the ``e.g.'', whose Latin expansion
  means ``for example''.  The other common abbreviation that we often
  use is ``i.e.'' whose Latin expansion means ``that is.''  Do not use
  i.e.  where e.g. is appropriate.

\item
  The word ``like'' should be replaced with ``such as''.

\item
  ``Softwares''  -- software is itself plural.

\item
  Data is both singular and plural.

\item
  ``less'' versus ``few'': Less is used for ``amorphous'' quantities
  (i.e., real numbers), few is used for discrete items (i.e.,
  integers).

\item
  ``From the top of head'' actually means unthinking.

\item
  ``Implemented'' does not mean ``used'' or ``installed''.

\item
  Latter == the second one, whereas later == (time wise, or ...)
  afterwards.
\end{enumerate}

I strongly urge you to read and heed the advice of the book
\cite{Strunk-and-White}.


% -eof-
