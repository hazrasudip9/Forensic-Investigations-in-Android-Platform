\chapter{Technical Report Writing}

A thesis is a technical report.  Read and follow a good style book
(e.g., Chicago) on technical report writing.  It is a good exercise to
mark {\em this} article up!  Additionally, every university has
guidelines regarding layouts and required pages such as a title page
an approval page.

(This article is itself poorly and hastily written.  Think of it as a
collection of notes.  ``Do as I say, not as I do! {\tt ;-)}'' )


\section{Order and Content of the Chapters}

This chapter is about the ordering of academic/technical content.  I
am particularly annoyed by the ``wrong order'' of chapters.  I
recommend the following order for the chapters.  But, do choose better
titles than the generic ones I am using below.

\begin{description}
\item[Abstract]

  Of course, Abstract is not a chapter!  In a technical report, the
  Abstract should be a summary of the contents.  Too many published
  papers include motivational sentences, and even citations.  These
  are inappropriate.  Typical length of a thesis abstract is a page.  

\item
  [Introduction] should restate the abstract, preferably using
  different sentences.  It is customary that the last paragraph
  describes the structure of the report.

\item
  [Background] describe concepts and terminology needed to understand
  the rest.  There should always be a chapter (in a thesis) or a
  section (in a paper) on backround.  This is a collection of
  definitions and (extremely brief) tutorials on what the reader
  should know in order to appreciate the main body of your tech
  report.  The length of this should never exceed, say, 1/10 of the
  tech report.  In a thesis, this is almost always the second chapter.

\item
  [Problem Statement] should be rigorous, and complete.  Also address
  the following: Why is it interesting to solve? Is it too trivial and
  not at the MTech/PhD level? Is it too hard to solve?  ``Problem
  Statement'' is a poor title -- choose better.

\item [Architecture and Design]
  Details of (Your) Solution. This may be multiple chapters.

\item [Related Work] There should always be a chapter (in a thesis) or
  section (in a paper) on related work.  This is an organized critique
  of the body of literature related to the work you are doing.  The
  expectation is that this is exhaustive.  What have others done?  Do
  summarize, but focus on critiquing.  This is not a collection of
  paragraphs summarizing individual papers.  It is almost always a bad
  idea to make this the second chapter.

  Place this chapter after a discussion of your work.  Then, you will
  be able to compare your work with that of others.    Discuss your
  dependencies -- what you are using from where?  This chapter (almost
  always) should follow your solution so that you can discuss
  comparatively your decisions in your solution.

\item
  [Discussion] of Alternatives (what else could have provided answers).

\item
  [Evaluation] of Your Solution (qualitative as well as quantitative).

\item
  [Conclusion] Use it with both of its meanings: deductions, and
  finishing.

\item
  [References] Must be ``properly'' done.  See the Section
  \ref{References}.


\end{description}

\section{Structure}

When you are moving from a chapter heading to its first section, and
from a section heading to its first subsection there is always an
introductory paragraph.

Sometimes we like to write a paragraph at the end of a chapter that
does not belong to the last section.  This is often done by
introducing some extra vertical white space after the end of the last
section that is longer than the typicial spacing between consecutive
sections.

There are no particular size requirements.  But, a one page chapter
and a two line section do look silly.

\section{URLs}

URLs are now common in CS theses.  Older style guides do not cover
these.  Include a URL inside \verb|\url{}|. Example:
\url{http://www.google.com/search?q=latex+eepic+pstricks+tikz}
The package named {\tt hyperref} has a macro named \verb|href| with
two arguments: first one, a URL, second one, the text to be displayed.


\section{Acronyms}

Acronyms are written in all-upper case.  Examples: {\sc gnu}, {\sc
  bcpl}, {\sc kde}.  Linux, Unix, Android, Scala, Java, ... are not
acronyms.  Do not write LINUX; change it to Linux, no boldface, L in
caps, the rest in lower case.

\section{References}
\label{References}

Use BiBTeX.  The file named {\tt thesis.bib} should contain the
references you collected in the {BibTeX} syntax.  I have included a
few references to illustrate this syntax; you can leave them in this
file.  In general, it can contain many more references than what you
cite in the body of your thesis.  The {\tt bibtex} program examines
the actual citations made, and pulls together the cited references
into the text file named {\tt thesis.bbl}.  See also Section
\ref{BibFiles}.

Pay attention to the complaints of {\tt bibtex}.  Publication venue
(conference or journal), year, and page numbers of the paper are a
must.  Include {\sc url} of the paper ({\sc html} or {\sc pdf}), if
available.

Bibliographic citations are done in many ways.  I prefer [Author-Name,
  Year] style.  This report used {\tt
  \verb|\bibliographystyle|\{Bib/acmtrans\}}, a style that
ACM\footnote{\url{http://www.acm.org/}} Transactions use.  Other
acceptable choices are those that include author(s) last name and year
of publication in the body of the text that cites, and the end of the
article list of references are sorted alphabetically by the first
author's last name.  Styles that produce a bracketed number, as in
[5], are not acceptable to me.

When using \url{http://scholar.google.com/}, set Bibliography Manager
to use BibTeX in the preferences.  Then, the references found
can be saved in \bibtex{} format.  Visit also
\url{https://www.mendeley.com/},
\url{http://bibsonomy.org/} and \url{http://zotero.org}.

\section{Plagiarism}

Using someoneelse's work without crediting them is plagiarism.  There
are many software tools and web services that can pinpoint lifted
sentences.  Quoting is ok, but must always use double- or sigle-quotes
and cite immediately.

% -eof-
